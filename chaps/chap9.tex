\chapter{SIIT提供的單詞表}
\section{Medical}
\subsection{Treatment \& Medications}
\begin{multicols}{2}
\begin{itemize}
  \itemsep0em
  \item analgesic: 止痛藥
  \item anaesthesia: 麻醉
  \item antibiotic: 抗生素
  \item antihistamine: 抗組胺藥\footnote{通常指$H_1$-受體拮抗劑,是一種,透過對體內$H_1$-受體(組織胺受體之一種)的作用,減少組織胺對這些受體產生效應,從而減輕身體對致敏原的過敏反應的藥物。}
  \item aspirin: 阿司匹林
  \item antiseptic solution: 防腐溶液
  \item bandage: 繃帶
  \item band-aid: 護創膠布
  \item bed-pan: 床上便盆
  \item blood transfusion: 輸血
  \item bracer: 護肘, 牙套
  \item caesarian section: 剖腹產
  \item capsule: 膠囊
  \item chemotherapy: 化療
  \item collar: 護頸
  \item contraception: 避孕, 節育
  \item condom: 避孕套避孕
  \item contraceptive: 避孕藥 / 工具
  \item conception: 懷孕 / 受孕
  \item cotton wool: 脫脂棉
  \item crutches: J字形拐杖
  \item go on crutches: 撐著拐杖走
  \item curette: 刮宮術
  \item denture: 假牙
  \item disinfectant: 消毒劑
  \item dosage: 劑量
  \item diuretics: 藥片
  \item dialysis: (血液)透析 / 洗胃
  \item dressing: 敷藥 / 包扎傷口
  \item dropper: 滴劑 / 滴管
  \item E.C.G (electrocardiogram): 心電圖
  \item E.E.G (electroencephalogram): 腦電圖
  \item enema: 灌腸劑(從肛門灌到大腸)\footnote{是指通過肛門引液體灌洗直腸的操作。有治療疾病(例如便秘)、另類保健療法、或者色情(例如性虐待)的用途。}
  \item external use: 外用
  \item extraction: 取出, 拔牙, 摘除
  \item eye drops / ointment: 眼藥水/膏
  \item eye patch: 眼罩
  \item family planning: 計劃生育
  \item filling: 填充物
  \item solid / fluid food: 固體食物 / 流食
\end{itemize}
\end{multicols}

\section{Medicare}
\begin{itemize}
  \itemsep0em
  \item Medicare: 國民醫療保健
  \item Bulk Bill: 刷Medicare卡公費醫療
  \item public / private patient: 公費醫療保險 / 私人醫療保險的病人
  \item Child Support: \hilight{子女撫養費}
  \item in patient / out patient: 住院病人 / 門診病人
  \item Pharmaceutical Benefits Scheme (PBS): 藥物補助計劃
  \item Repatriation Pharmaceutical Benefits Scheme: 退伍軍人藥物補助計劃
  \item Health Care Card: 健康醫療保健卡 (一般用於低收入或上年紀的人)
  \item PBS Safety Nets: 藥物補助計劃安全網 (檢查哪些藥物項目被cover)
  \item Teen Dental: 青少年牙科服務
  \item Make a Claim: 報銷申請
  \item Commonwealth Seniors Health Card: 聯邦老年保健卡
  \item Office of Hearing Services: 聽力服務處
  \item Medicare Benefit Tax Statement: 國民醫療保健稅務報告
  \item Medicare Benefits Schedule (MBS): 國民醫療保健福利計劃
  \item Medicare Levy Exemption: 國民醫療保健豁免
  \item Health Identifiers Service: 醫療保健尋找服務 (適用於偏遠地區的人)
  \item \hilight{Cleft Lip} and \hilight{Cleft Palate Scheme}: 唇齶裂畸形計劃
  \item Australian Government Department of Human Service: 澳洲\hilight{民政部}
\end{itemize}

\section{Centrelink}
\begin{multicols}{2}
\begin{itemize}
  \itemsep0em
  \item Acceptable proof of Identity (POI): 認可的身份證明文件
  \item Access points: 代辦處, 代理處
  \item Activity test: 尋工活動評估\footnote{當申請津貼的時候, 申請人需要證明自己有努力積極地在找工作.}
  \item Administrative Appeals Tribunal(AAT): 行政事務上訴仲裁庭
  \item Approved care: 核准的托兒服務
  \item Approved course of study: 核准的學習課程
  \item welfare agency: 福利機構
  \item assessable income: 應評估收入
  \item assets disqualifying limit: (可領取福利金的)財產限額
  \item award: 勞資裁定協議
  \item bulk billing\footnote{A payment option under the Medicare. It can cover a prescribed range of health services as listed in the Medicare Benefits Schedule, at the discretion of the health service provider.}: 公費醫療 / 保險報銷醫療
  \item capacity of work: 工作能力
  \item carer payment: 照顧者收入補貼
  \item carer allowance: 照顧者津貼
  \item casual earnings: 非固定收入
  \item reference number: 客戶號碼
  \item certified copy: 認可的副本
  \item child support: 兒童撫養費
  \item custodial parent: 有監護權的父母
  \item damages: 賠償金
  \item disability support pension: 殘疾人福利金
  \item double orphan pension: 雙重孤兒撫養津貼\footnote{申領資格: \url{https://www.humanservices.gov.au/customer/services/centrelink/double-orphan-pension}}
  \item stood down: 停職、停工
  \item undisclosed income: 未申報的收入
  \item worker's compensation: 工傷賠償、勞工賠償
  \item meals-on-wheels: 流動送餐服務
  \item entitlement: 應得金額 / 權利
  \item exempt income: 免徵稅收入
  \item Family Assistance Office (FAO): 家庭輔助處
  \item inability to work: 喪失工作能力
  \item income statement: 收入證明
  \item lump sum payment: 一次性付款
  \item pay slip: 工資單
  \item lump sum advance: 一次性預付
  \item maternity allowance: 產假津貼
  \item maternity immunisation allowance: 嬰兒免疫津貼
  \item means test: 收入資產評估
  \item medical certificate: 醫療證明
  \item mobility allowance: 行動不便者津貼
  \item naturalisation certificate: 入籍證書
  \item Unemployment allowance / benefit: 失業補貼
  \item Newstart allowance: 新開始津貼\footnote{為那些正在尋找工作的人士提供的輔助收入。您需要符合下列條件才能申請:年齡20以上。失業但是能夠工作,同時\\也在積極找工;另外已經在就近的 Centrelink 辦事處註冊登記。}
  \item Youth allowance: 青年津貼(小於25歲)\footnote{為青年提供的一項新的津貼。在生病、尋工、學習或接受訓練期間,您都可以申請領取這項福利。}
  \item nursing home: 照料中心
  \item parenting payment: 家長補助金
  \item pharmaceutical allowance: 藥品津貼
  \item primary earner: 主要收入賺取者
  \item refugee status: 難民身份
  \item severance pay: 遣散費
  \item Social Security Appeal Tribunal (SSAT): 社會保障上訴仲裁庭
  \item superannuation: 公積金, 退休金
  \item unfit for work: 不適於工作
  \item family day-care centre: 家庭日托中心
  \item labour market: 勞動力市場
  \item eligibility: (申請)資格
\end{itemize}
\end{multicols}