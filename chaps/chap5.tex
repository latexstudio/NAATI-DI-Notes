\chapter{雜類話題II}
\section{2016年2月29日 (Instructor: Vivian)}
\subsection{立遺囑}
\subsubsection*{需要掌握的單詞短語}
\begin{multicols}{2}
\begin{itemize}
  \itemsep0em
  \item 遺漏, 錯過機會: miss out
  \item 房產, 財產: property\footnote{property這個詞需要根據上下文來確定如何進行翻譯!}
  \item 資產: assets
  \item 遺產: estate
  \item 廢除: \hilight{revoke}
  \item 所有物: possessions
  \item 佔有欲: possessive
  \item 銀行存款: bank savings
  \item 自己存的退休金: \hilight{superannuation}
  \item 政府撫卹發放的養老金: pension
  \item 免除, \hilight{一筆勾銷}: be forgiven
  \item 被繼承: be inherited
  \item 健在的家庭成員: surviving family member
  \item 財產的繼承人: heirs to the property
  \item 個人債務: personal debts
  \item 值得信賴的: \hilight{trustworthy}
  \item 個人情況: personal circumstances
\end{itemize}
\end{multicols}

\subsubsection*{需要掌握的句型}
\begin{itemize}
  \itemsep0em
  \item 我想知道: I wander... (盡量少用, 這個表請求的語氣)
  \item 問區別:
  \begin{itemize}
  \itemsep0em
  	\item Is there a difference between doing A and (doing) B?
  	\item Does it make a difference if I do A or do B?
  	\item ...if one dies with or without a will?
  \end{itemize}
  \item 我的遺囑什麼時候可以生效:
  \begin{itemize}
  \itemsep0em
  	\item How can my will come into effect?
  	\item How can I make my will come into valid?
  \end{itemize}
  \item 遺囑里應該包含什麼:
  \begin{itemize}
  \itemsep0em
  	\item What should I include in my will?
  	\item What should be included in a will?
  \end{itemize}
  \item 通過遺囑檢驗獲批來生效: put into effect by a grant of probate.
  \item 以書面, 有簽字並見證的(方式): be in writing, signed and witnessed.
  \item 我對...有...的股份: of which / in which I hold...of the shares.
  \item 可能有責任承擔(法律責任): may be (held) liable for $sth.$ / to do...
  \item \hilight{貸款: take out a loan / mortgage}
  \item \hilight{資金週轉(不靈): cash / capital flow (difficulties)}
  \item 這樣是為了...: so as to do...
  \item 考慮做某事: consider doing...
  \item 對我有幫助: helpful to me
  \item 幫了我大忙了:
  \begin{itemize}
  \itemsep0em
  	\item It has helped me a lot!
  	\item It has been really helpful!
  \end{itemize}
\end{itemize}

\subsection{減肥}
\subsubsection*{需要掌握的單詞短語}
\begin{multicols}{2}
\begin{itemize}
  \itemsep0em
  \item 迭縫帶環: lap-band (surgery)\footnote{an inflatable silicone device placed around the top portion of the stomach to treat obesity, intended to slow consumption of food and thus reduce the amount of food consumed.}
  \item 脂肪抽吸手術, 抽脂術: liposuction\footnote{Areas affected can range from the abdomen, thighs and buttocks, to the neck, backs of the arms and elsewhere.}
  \item 肥胖症患者: \hilight{obese patient}
  \item 新陳代謝: \hilight{metabolism}
  \item 減肥藥: weight-loss pill
  \item 便秘的: constipated
  \item 壓迫: stain
  \item 減肥: \hilight{shape up / slim down}
  \item 高膽固醇: high \hilight{cholesterol}
  \item 高脂肪食物: \hilight{fatty foods}
  \item 心絞痛: angina
  \item 蔬菜和水果: fruit and vegetable\footnote{在英語中蔬菜水果的位置要顛倒}
\end{itemize}
\end{multicols}

\subsubsection*{需要掌握的句型}
\begin{itemize}
  \itemsep0em
  \item 我覺得疼
  \begin{itemize}
  \itemsep0em
  	\item I am \hilight{in pain}.\footnote{建議用這個, 既可以表示生理, 也可以表示心理.}
  	\item I feel pain in...
  	\item There is a pain in...
  \end{itemize}
  \item 疼得很厲害
  \begin{itemize}
  \itemsep0em
  	\item \hilight{unbearable pain}
  	\item Pain is killing me.
  \end{itemize}
  \item 促進新陳代謝: \hilight{increase metabolism}
  \item 負擔不起風險: \hilight{afford} to take such risk
  \item 一一對應: \hilight{match $sth.$ with $sth.$}
  \item 不介意做: I don't mind doing $sth.$
  \item 我只好: I have no choice but...
  \item 節食: be on a diet
\end{itemize}

\subsection{酒駕}
\subsubsection*{需要掌握的單詞短語}
\begin{multicols}{3}
\begin{itemize}
  \itemsep0em
  \item 零點零七: 0.07 / \hilight{.07}\footnote{在對話中可能.之前的零不會讀出來, 要格外注意!}
  \item 讀數: reading
  \item 初犯: \hilight{first time offender}
  \item 從輕處罰: reduce the punishment
  \item 罰我錢: fine me\footnote{fine在這裡做$v.$}
  \item 罰單: penalty notice
  \item 照章辦事: follow the rule
  \item 嚴厲: \hilight{harsh}\footnote{可以和punishment搭配}
  \item 沒氣, 沒電: flat
\end{itemize}
\end{multicols}

\subsubsection*{需要掌握的句型}
\begin{multicols}{2}
\begin{itemize}
  \itemsep0em
  \item 為什麼只攔下我呢: Why did you only stop me?
  \item 勸人喝酒: persuade / make $sb.$ to drink
  \item \hilight{屬於是}: falls into / within...
  \item 電池沒電了:
  \begin{itemize}
  \itemsep0em
  	\item The battery is dying.
  	\item The battery is flat.
  	\item I have run out of battery.
  \end{itemize}
\end{itemize}
\end{multicols}

\subsection{骨質疏鬆症}
\subsubsection*{需要掌握的單詞短語}
\begin{multicols}{2}
\begin{itemize}
  \itemsep0em
  \item 骨質疏鬆症: osteoporosis\footnote{骨質疏鬆症是一種鈣質由骨骼往血液淨移動的礦物質流失(demineralization)現象,骨質量減少,骨骼內孔隙增大,呈現中空疏鬆現象,速率取決於破骨細胞(osteoclast)和成骨細胞(osteoblast)活性的消長。此需和軟骨症(osteomalacia)有所區別,軟骨症的成因是維生素D的缺乏所導致。}
  \item 腰疼: lumbago = lower back pain
  \item 腰椎間盤: lumbar discs
  \item ...誘發的: ...-induced
  \item 脊椎骨裂: fractures\footnote{這裡不要翻譯成骨折, 否則顯得很嚴重} in the spine
  \item 無症狀的: asymptomatic
  \item 止痛貼: analgesic patch
\end{itemize}
\end{multicols}

\subsubsection*{需要掌握的句型}
\begin{multicols}{2}
\begin{itemize}
  \itemsep0em
  \item 腰疼還是老樣子: lower back pain \hilight{remains the same}.
  \item 讓我做...測試: order...test for me
  \item 更容易患上...: be more susceptible to $sth.$
  \item 怎麼治: What treatments are needed?
  \item 藥一起吃: medications \hilight{mix} together.
\end{itemize}
\end{multicols}

\subsection{車禍}
\subsubsection*{需要掌握的單詞短語}
\begin{multicols}{2}
\begin{itemize}
  \itemsep0em
  \item \hilight{大清早}: early hours
  \item 左前方: \hilight{front-left}
  \item 轉彎: turn the corner
  \item 恢復知覺: \hilight{regain} your consciousness
  \item 方向盤: steering wheel
  \item 電線桿: (power) pole
  \item 行人: \hilight{pedestrians}
  \item 發生: take place
  \item (事故)現場: \hilight{scene}
  \item \hilight{證物}: exhibit
  \item 速度標誌: speed advisory sign
\end{itemize}
\end{multicols}

\subsubsection*{需要掌握的句型}
\begin{multicols}{2}
\begin{itemize}
  \itemsep0em
  \item 以...的速度行車: \hilight{drive (at the speed) of...}
  \item 沒有用: it was no use
  \item 我沒有什麼可辯解的了: I have nothing (further) to add in my defence.
\end{itemize}
\end{multicols}

\subsection{賣房}
\mybox{\centering \textbf{注意}: 更多內容請見詞彙專題里的和競拍有關的詞!}
\subsubsection*{需要掌握的單詞短語}
\begin{multicols}{2}
\begin{itemize}
  \itemsep0em
  \item 迅速上升: spike up
  \item 在你名下: \hilight{under your name}
  \item 房產證, 房契, 地契: title deed\footnote{中國的房產證可以翻譯成property ownership certificate}
  \item 菜地: veggie patch
  \item 噴泉: fountain
\end{itemize}
\end{multicols}

\subsubsection*{需要掌握的句型}
\begin{multicols}{2}
\begin{itemize}
  \itemsep0em
  \item 把它賣個好價: sell it for a high price.
  \item 把...放到拍賣: \hilight{put it up for} auction.
  \item 你認為需要多久: How long do you think will it take to...?
\end{itemize}
\end{multicols}

\section{2016年3月1日 (Instructor: Chris)}

\vspace{15mm}
\begin{center}
  \textbf{************ Stay tuned ... ************}
\end{center}