\chapter{NAATI口譯考試基本介紹}
\section{基本概念}
\begin{itemize}
  \itemsep0em
  \item NAATI Accredited \hilight{Para}professional - NAATI認證的\hilight{輔助}口譯員
  \item 一般一篇文章會有300左右的詞數,每個segment一般小於35詞, 如果沒聽清可以ask for repeat(一篇文章只有\hilight{一次不扣分}的要求重讀的機會)
  \item 口譯證書的有效期為三年, 可由單位開具證明說明證書持有者從事翻譯工作和活動, 否則證書在三年到期作廢.
  \item Sight Translation - \textbf{視譯}\footnote{所謂視譯,就是看著中文稿不間斷地口頭翻譯成英文或反過來將英文譯成中文。視譯是同聲傳譯中最常用的訓練方法 \\ 之一。視譯練習不僅越來越多地被用於交替傳譯的培訓,同時也是漢語主導環境條件下練習口語的有效方法。}, 是平時用來訓練口譯的其中一種辦法.
\end{itemize}

\section{關於老師}
\begin{itemize}
  \itemsep0em
  \item \textbf{Trainer}: Vivian Ma (Head trainer, 週一上課), Chris Quan (週二上課)
  \item \textbf{Coordinator}: Jiali Liu - 負責發送Mock Exam Confirmation
\end{itemize}

\section{通過NAATI考試以後如何加PR的5分}
還需要給NAATI遞交: Accreditation Application, Diploma Certificate, Assignment Booklet (自己錄音完成) 和 Recommendation Letter (由SIIT開出).

\section{二级口译的几大話題}
Medical, Legal, \hilight{CentreLink(福利署)}, Education, Housing, Insurance, Finance, Business, Investment.

\section{一些基本方法和技巧}
\begin{itemize}
  \itemsep0em
  \item 聽錄音時合理地適當猜測上下文可能出現的內容, 在開口之前一定想好句子的時態和詞語的\hilight{時態}.
  \item 在翻譯過程中要壓制住自己的猶豫聲, 例如``啊...額...", 不要在翻譯途中說``sorry".
  \item 不要回頭重說(backtrack). 若是有遺漏重要的信息, 可另起一句話補足意思.
  \item 表達要流暢, 不要添加不必要的口頭禪.
  \item 20\%的note-taking, 80\%的理解 + 短時記憶, 不要一直悶頭做筆記.
  \item 培養出適合自己的一套符號用於快速筆記: 比如:
    \mybox{\centering GM/GA/GE, $\surd$, $\ge$, $\ll$, $\in$, $\not=$, $\Delta$, $\nearrow$, $\swarrow$, $\hookrightarrow$, $impro^{ed}$, $\heartsuit$.}
    \item 好的口譯筆記需要達到``變廢為寶" - 前文提到的內容在後文被重復提到的話, 可以用箭頭下拉而不是再寫一遍.
\end{itemize}